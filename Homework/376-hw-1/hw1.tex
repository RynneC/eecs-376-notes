\documentclass[11pt,addpoints]{exam}
\usepackage{fullpage}
\usepackage{amsfonts, amsmath, amssymb, amsthm}
\usepackage{mathtools}
\usepackage{enumerate}
\usepackage{array}
\usepackage{amssymb}
\usepackage{listings}
\usepackage[boxed]{algorithm}
\usepackage[noend]{algpseudocode}
\usepackage{hyperref}

% in order to compile this file you need to get 'header.tex' from
% Canvas and change the line below to the appropriate file path
%%% theorems

\theoremstyle{plain}            % following are "theorem" style
\newtheorem{theorem}{Theorem}[section]
\newtheorem{lemma}[theorem]{Lemma}
\newtheorem{corollary}[theorem]{Corollary}
\newtheorem{proposition}[theorem]{Proposition}
\newtheorem{claim}[theorem]{Claim}
\newtheorem{fact}[theorem]{Fact}
\newtheorem{openproblem}[theorem]{Open Problem}

\theoremstyle{definition}       % following are def style
\newtheorem{definition}[theorem]{Definition}
\newtheorem{conjecture}[theorem]{Conjecture}
\newtheorem{example}[theorem]{Example}
\newtheorem{protocol}[theorem]{Protocol}
\newtheorem{exercise}[theorem]{Exercise}

\theoremstyle{remark}           % following are remark style
\newtheorem{remark}[theorem]{Remark}
\newtheorem{note}[theorem]{Note}
\newtheorem*{aside}{Aside}
\newtheorem*{gn}{Grading Note} % grading note

%%% special sets
\newcommand{\bit}{\ensuremath{\{0,1\}}}
\newcommand{\bitt}{\ensuremath{\{-1,1\}}}

% font for general-purpose algorithms
\newcommand{\algo}[1]{\ensuremath{\mathsf{#1}}}
% font for general-purpose computational problems
\newcommand{\problem}[1]{\ensuremath{\mathsf{#1}}}
% font for complexity classes
%\newcommand{\class}[1]{\ensuremath{\mathsf{#1}}}

\global\long\def\path{\mathrm{path}}
% asymptotics
%\DeclareMathOperator{\poly}{poly}
%\DeclareMathOperator{\polylog}{polylog}
\DeclareMathOperator{\negl}{negl}
\DeclareMathOperator{\Otil}{\tilde{O}}
\DeclareMathOperator{\Ostar}{O^*}

%%% "LEFT-RIGHT" PAIRS OF SYMBOLS

% inner product
\DeclarePairedDelimiter\inner{\langle}{\rangle}
% absolute value
\DeclarePairedDelimiter\abs{\lvert}{\rvert}
% a set
\DeclarePairedDelimiter\set{\{}{\}}
% parens
\DeclarePairedDelimiter\parens{(}{)}
% tuple, alias for parens
\DeclarePairedDelimiter\tuple{(}{)}
% square brackets
\DeclarePairedDelimiter\bracks{[}{]}
% rounding off
\DeclarePairedDelimiter\round{\lfloor}{\rceil}
% floor function
\DeclarePairedDelimiter\floor{\lfloor}{\rfloor}
% ceiling function
\DeclarePairedDelimiter\ceil{\lceil}{\rceil}
% length of some vector, element
\DeclarePairedDelimiter\length{\lVert}{\rVert}
% norm
\DeclarePairedDelimiter\norm{\lVert}{\rVert}
% length (as a string)
\DeclarePairedDelimiter\len{\lvert}{\rvert}
% bra-kets
\DeclarePairedDelimiter\bra{\langle}{\rvert}
\DeclarePairedDelimiter\ket{\lvert}{\rangle}
\newcommand{\braket}[2]{\ensuremath{\langle #1 \vert #2 \rangle}}
\newcommand{\ketbra}[2]{\ensuremath{\lvert #1 \rangle \langle #2 \rvert}}

%%% MATH OPERATORS
\DeclareMathOperator{\Var}{Var}
\DeclareMathOperator{\tr}{Tr}
\DeclareMathOperator{\supp}{supp}
\DeclareMathOperator{\var}{var}
\DeclareMathOperator{\vol}{vol}
\DeclareMathOperator{\sign}{sign}

%%% BLACKBOARD SYMBOLS

\newcommand{\C}{\ensuremath{\mathbb{C}}}
\newcommand{\D}{\ensuremath{\mathbb{D}}}
\newcommand{\F}{\ensuremath{\mathbb{F}}}
\newcommand{\G}{\ensuremath{\mathbb{G}}}
\newcommand{\J}{\ensuremath{\mathbb{J}}}
\newcommand{\N}{\ensuremath{\mathbb{N}}}
\newcommand{\Q}{\ensuremath{\mathbb{Q}}}
\newcommand{\R}{\ensuremath{\mathbb{R}}}
\newcommand{\T}{\ensuremath{\mathbb{T}}}
\newcommand{\Z}{\ensuremath{\mathbb{Z}}}
\newcommand{\QR}{\ensuremath{\mathbb{QR}}}

% sets in calligraphic type

\newcommand{\calD}{\ensuremath{\mathcal{D}}}
\newcommand{\calF}{\ensuremath{\mathcal{F}}}
\newcommand{\calG}{\ensuremath{\mathcal{G}}}
\newcommand{\calH}{\ensuremath{\mathcal{H}}}
\newcommand{\calI}{\ensuremath{\mathcal{I}}}
\newcommand{\calL}{\ensuremath{\mathcal{L}}}
\newcommand{\calN}{\ensuremath{\mathcal{N}}}
\newcommand{\calP}{\ensuremath{\mathcal{P}}}
\newcommand{\calS}{\ensuremath{\mathcal{S}}}
\newcommand{\calX}{\ensuremath{\mathcal{X}}}
\newcommand{\calY}{\ensuremath{\mathcal{Y}}}

% matrices and vectors

\newcommand{\matA}{\ensuremath{\mathbf{A}}}
\newcommand{\matB}{\ensuremath{\mathbf{B}}}
\newcommand{\matC}{\ensuremath{\mathbf{C}}}
\newcommand{\matD}{\ensuremath{\mathbf{D}}}
\newcommand{\matE}{\ensuremath{\mathbf{E}}}
\newcommand{\matF}{\ensuremath{\mathbf{F}}}
\newcommand{\matG}{\ensuremath{\mathbf{G}}}
\newcommand{\matH}{\ensuremath{\mathbf{H}}}
\newcommand{\matI}{\ensuremath{\mathbf{I}}}
\newcommand{\matJ}{\ensuremath{\mathbf{J}}}
\newcommand{\matK}{\ensuremath{\mathbf{K}}}
\newcommand{\matL}{\ensuremath{\mathbf{L}}}
\newcommand{\matM}{\ensuremath{\mathbf{M}}}
\newcommand{\matN}{\ensuremath{\mathbf{N}}}
\newcommand{\matO}{\ensuremath{\mathbf{O}}}
\newcommand{\matP}{\ensuremath{\mathbf{P}}}
\newcommand{\matQ}{\ensuremath{\mathbf{Q}}}
\newcommand{\matR}{\ensuremath{\mathbf{R}}}
\newcommand{\matS}{\ensuremath{\mathbf{S}}}
\newcommand{\matT}{\ensuremath{\mathbf{T}}}
\newcommand{\matU}{\ensuremath{\mathbf{U}}}
\newcommand{\matV}{\ensuremath{\mathbf{V}}}
\newcommand{\matW}{\ensuremath{\mathbf{W}}}
\newcommand{\matX}{\ensuremath{\mathbf{X}}}
\newcommand{\matY}{\ensuremath{\mathbf{Y}}}
\newcommand{\matZ}{\ensuremath{\mathbf{Z}}}
\newcommand{\matzero}{\ensuremath{\mathbf{0}}}

\newcommand{\veca}{\ensuremath{\mathbf{a}}}
\newcommand{\vecb}{\ensuremath{\mathbf{b}}}
\newcommand{\vecc}{\ensuremath{\mathbf{c}}}
\newcommand{\vecd}{\ensuremath{\mathbf{d}}}
\newcommand{\vece}{\ensuremath{\mathbf{e}}}
\newcommand{\vecf}{\ensuremath{\mathbf{f}}}
\newcommand{\vecg}{\ensuremath{\mathbf{g}}}
\newcommand{\vech}{\ensuremath{\mathbf{h}}}
\newcommand{\veck}{\ensuremath{\mathbf{k}}}
\newcommand{\vecm}{\ensuremath{\mathbf{m}}}
\newcommand{\vecp}{\ensuremath{\mathbf{p}}}
\newcommand{\vecq}{\ensuremath{\mathbf{q}}}
\newcommand{\vecr}{\ensuremath{\mathbf{r}}}
\newcommand{\vecs}{\ensuremath{\mathbf{s}}}
\newcommand{\vect}{\ensuremath{\mathbf{t}}}
\newcommand{\vecu}{\ensuremath{\mathbf{u}}}
\newcommand{\vecv}{\ensuremath{\mathbf{v}}}
\newcommand{\vecw}{\ensuremath{\mathbf{w}}}
\newcommand{\vecx}{\ensuremath{\mathbf{x}}}
\newcommand{\vecy}{\ensuremath{\mathbf{y}}}
\newcommand{\vecz}{\ensuremath{\mathbf{z}}}
\newcommand{\veczero}{\ensuremath{\mathbf{0}}}
\newcommand{\vecone}{\ensuremath{\mathbf{1}}}

%%% misc

\newcommand{\eps}{\varepsilon}

%%% for homework and section notes

\newcommand{\commonheader}[2]{
    \pagestyle{headandfoot}
    \setlength{\headheight}{26pt}
    \setlength{\headsep}{16pt}

    \header
        {\small{\textbf{EECS 376: Foundations of Computer Science}} \\ \footnotesize{\textbf{University of Michigan, Spring 2024}}}
        {#1}
        {#2}

    \firstpageheadrule
    \runningheadrule

    \footer
        {}
        {\thepage}
        {}
}

\newcommand{\hwheader}{
    \commonheader
        {\Large \textbf{Homework \hwnum}}
        {\small \textbf{Due 8:00pm, \duedate\\ {\tiny(accepted until 9:59 pm, no credit after)}}}
}

\newcommand{\hwslnheader}{
    \commonheader
    	{}
        {\Large \textbf{Solutions to Homework \hwnum}}
    \printanswers
}

\newcommand{\notesheader}{
    \commonheader
    	{}
        {\Large \textbf{Discussion Notes \sectionnum}}
}

\newcommand{\practiceheader}{
    \commonheader
    	{}
        {\Large \textbf{Discussion Worksheet \sectionnum}}
}

\newcommand{\practiceslnheader}{
    \commonheader
    	{}
        {\Large \textbf{Solutions to Discussion Worksheet \sectionnum}}
}

\newcommand{\reviewheader}{
    \commonheader 
    \smallskip
    	{}
        {\Large \textbf{Midterm Review Notes}}
}

\newcommand{\hwpreface}{

\noindent This homework has \numquestions\ questions, for a total of \numpoints\ points and \numbonuspoints\ extra-credit points.

\noindent Unless otherwise stated, each question requires \emph{clear}, \emph{logically correct}, and \emph{sufficient} justification to convince the reader.

\noindent For bonus/extra-credit questions, we will provide very limited guidance in office hours and on Piazza, and we do not guarantee anything about the difficulty of these questions.
 
\noindent We strongly encourage you to typeset your solutions in \LaTeX.

\noindent If you collaborated with someone, you must state their name(s). You must \emph{write your own solution} for all problems and \emph{may not use any other student’s write-up}.
}

\newcommand{\hint}[1]{
\emph{Hint}: #1
}
\newcommand{\bonus}{
\emph{Optional bonus}:
}
\newcommand{\extracredit}{
\emph{Extra credit}: 
}


\newcommand{\hwnum}{1}
\newcommand{\duedate}{May 16}
\usepackage{xcolor}

% remove "draft" to turn off fixme notes
\usepackage[draft,multiuser,inline,nomargin]{fixme}
% fixme notes: register commands for author(s)
\FXRegisterAuthor{a}{ea}{\color{red}Ali}
% add more HW authors here...
\FXRegisterAuthor{j}{ej}{\color{blue}Junghwan}
\FXRegisterAuthor{e}{ee}{\color{brown}Eric}
\fxusetheme{color}

\hwheader   % header for homework
% \hwslnheader   % header for homework solutions

% Comment the following line in order to hide solutions.
% Uncomment the line to show solutions written inside of
% LaTeX solution environments like:
%   \begin{solution}
%     My solution.
%   \end{solution}.
\printanswers

\begin{document}

\listoffixmes

\hwpreface

\pointsinmargin
\pointpoints{pt}{pts}
\bonuspointpoints{EC pt}{EC pts}
\marginpointname{ \points}
\marginbonuspointname{ \bonuspoints}

\begin{questions}

  \addtocounter{question}{-1}
  \question[0] \textbf{Before you start; before you submit.}
  
  \begin{parts}
    \part Carefully read \href{https://drive.google.com/file/d/1Q6E2B2hljH28vv7L1LZOugiJol-t1khq/view?usp=drive_link}{Handout 1} before starting this assignment, and apply it to the solutions you submit.
    
    \part If applicable, state the name(s) and uniqname(s) of your collaborator(s).

    \begin{solution}
    None.
       
    \end{solution}
  \end{parts}

  \question \textbf{Practice with asymptotics (``big-Oh, big-Omega, big-Theta'').}
  
  For the following pairs of functions, state with justification whether or not each of the following hold: $f(n) = O(g(n))$, $f(n) = \Omega(g(n))$, $f(n) = \Theta(g(n))$.

  You can find the definitions of asymptotic notations ($O, \Omega, \Theta$) in \href{https://drive.google.com/file/d/1F_6pnQxpaYiqpnue5pKDSYVMItf3PG-T/view?usp=drive_link}{Handout 0}.

  \begin{parts}
    \part[5] $f(n) = n^3 + 2n + 8$, $g(n) = 4n^3$.
    
    \begin{solution}
    \\
    1. $f(n) = O(g(n))$.
    \\Consider $c = 1$ and $n_0 = 2$, then $cg(n) -f(n)= n(3n^2-2) - 8.$ When $n=n_0$,   $cg(n) -f(n)= 20 - 8 = 12.$ Since $cg(n) -f(n)$ increases as $n$ grows when $3n^2-2 > 0$, $f(n) \leq cg(n)$ when $n \geq n_0$. Therefore $f(n) = O(g(n))$. \\ 
    \\2. $f(n) = \Omega(g(n))$.
    \\Consider $c = {1\over 4}$ and $n_0 = 1$, then $f(n) - cg(n)= n^3 + 2n + 8 - n^3 = 2n + 8.$ When $n >= 1$, $f(n) - cg(n)>0$, $f(n) > cg(n)$. Therefore $f(n) = \Omega(g(n))$\\
    \\3. $f(n) = \Theta(g(n))$.
    \\Since $f(n) = O(g(n))$ and $f(n) = \Omega(g(n))$, $f(n) = \Theta(g(n))$ by definition.
    \end{solution}
    
    \part[5] $f(n) = (2 + (-1)^n) n^2 + 1$, $g(n) = n^2$.
    
    \begin{solution}
    \\  1. $f(n) = O(g(n))$.
    \\Consider $c = 4$ and $n_0 = 1$, \(f(x) -cg(x) = (-1)^n n^2 - 2n^2 + 1 \leq -n^2 + 1\). So when $n \geq 1$, $-n^2 +1 \leq 0$, $f(n) \leq cg(n)$. Therefore $f(n) = O(g(n))$.
    \\\\2. $f(n) = \Omega(g(n))$.
    \\Consider $c = 1$ and $n_0 = 1$, \(f(x) -cg(x) = (-1)^n n^2 + n^2 + 1 \geq 1\). So $f(n) \geq cg(n)$. Therefore $f(n) = \Omega(g(n))$.
    \\\\3. $f(n) = \Theta(g(n))$.
    \\Since $f(n) = O(g(n))$ and $f(n) = \Omega(g(n))$, $f(n) = \Theta(g(n))$ by definition.
    \end{solution}

    \part[5] $f(n) = \log_{2}(n^{10})$, $g(n) = \log_{10}(n^{2})$.

    \begin{solution}
    \(f(x) = 10\log_2(n)\),
    \(g(x) = 2\log_{10}(n) = 2\frac{\log_2(n)}{\log_2(10)} = \frac{2}{\log_2(10)}\log_2(n)\)
    \\Therefore $f(n) = \Omega(g(n))$, $f(n) = O(g(n))$ and $f(n) = \Theta(g(n))$ since we can choose $c = 5\log_2(10)$ to let $f(x) = cg(x)$.
    \end{solution}

    \part[5] $f(n) = 2^{10n}$, $g(n) = n^{10} \cdot 10^{2n}$.

    \begin{solution}
    \(f(x) = 2^{10n} = (2^{10})^n = 1024^n\),  \(g(x) = n^{10} \cdot 10^{2n} = n^{10} \cdot (10^2)^n = n^{10} \cdot 100^n\).\\
    $$
    \lim_{n \to \infty} \frac{g(n)}{f(n)} = 
    \lim_{n \to \infty} \frac{n^{10} \cdot 100^n}{1024^n} = 
    \lim_{n \to \infty} (\frac{25}{256})^n \cdot n^{10} =
    \lim_{n \to \infty} \frac{n^{10}}{(\frac{256}{25})^n} = 0
    $$ by the L'Hôpital's Rule. 
    \\Therefore $f(n) \not= O(g(n))$, $f(n) = \Omega(g(n))$.
    \\ And $f(n) \not= \Theta(g(n))$ since not both expressions are true.
    \end{solution}

  \end{parts}

  \question[5] \textbf{Comparing asymptotic running times.}

  Suppose that Algorithm X has running time $T_{X}(n) = \Theta(n)$, Algorithm Y has running time $T_{Y}(n) = \Theta(n^2)$, and Algorithm Z has running time $T_{Z}(n) = O(n^{2} \log n)$.

  As usual, all running times are stated in terms of the \emph{worst case} for inputs of size~$n$.
  That is, $T_{X}(n)$ is the \emph{maximum} number of steps for which X runs, taken over all inputs of size~$n$ (and similarly for $T_{Y}, T_{Z}$).
  Choose all claims that are necessarily true.

\checkboxchar{\raisebox{-.2em}{\Large$\square$}}
\checkedchar{\raisebox{-.2em}{\Large$\blacksquare$}}
  \begin{checkboxes}
    \choice On every input, X runs faster than Y.

    \choice For all large enough $n$, X runs faster than Y on every input of size $n$.

    \CorrectChoice For all large enough $n$, there is an input of size $n$ for which X runs faster than Y.

    \CorrectChoice For all large enough $n$, there is an input of size $n$ for which Y runs faster than Z.
  \end{checkboxes}


  \question \textbf{EECS 376 lover.}

  Consider the following algorithm:

\begin{minipage}{\linewidth}
\begin{algorithm}[H]
      \begin{algorithmic}[1]
        \Function{\text{EECS376Lover}}{$n, k$} \Comment{$n$ is a positive integer, and $k \in \set{1, 2, \ldots, n}$}
        \For{$i = 1, 2, \dots, k$}
            \For{$j = 1, 2, \dots, n-k$}            
            \State {\Call{Print}{\texttt{"I love EECS 376!"}}}
            \EndFor
        \EndFor
        \EndFunction
      \end{algorithmic}
\end{algorithm}
\end{minipage}

\begin{parts}
    \part[1] Identify the value of $k$ that induces the \textbf{most} \texttt{"I love EECS 376!"} printed by the algorithm. 

    \part[2] Based on your answer in (a), give the \textbf{tightest correct asymptotic} (big-$O$) bound, as a function of~$n$, on the number of \texttt{"I love EECS 376!"} printed by the algorithm. 

    \part[2] Is the algorithm \emph{efficient}, i.e., runs in at most polynomial time with respect to the input size? Briefly explain your answer.
\end{parts}

\begin{solution}
  \begin{parts}
    \part 
    Fix $n$, for any $1 \leq k \leq n$, the number of \texttt{"I love EECS 376!"} printed by the algorithm is $k(n-k)$ = $-(k-{n\over 2})^2 + {n^2 \over 4}$.\\
    Therefore the maximum is reached when $k$ is closest to ${n\over 2}$.\\
    So the value of $k$ that induces the most \texttt{"I love EECS 376!"} printed by the algorithm is $k = \lfloor {n\over 2} \rfloor$ and $k = \lceil {n\over 2} \rceil$. (If $n$ is even, it is just $n\over 2$; if $n$ is odd, then it is ${n\pm 1}\over 2$)
    
    \part 
    Based on the answer in (a), the tightest correct asymptotic bound is $O({{n^2}})$ since the number of \texttt{"I love EECS 376!"} printed = $k(n-k)$ = $-(k-{n\over 2})^2 + {n^2 \over 4} \leq {n^2 \over 4}$
    
    \part 
    The algorithm is efficient since ts running time is within $O(n^2)$, which is polynomial time with respect to the input size.
\end{parts}
\end{solution}

  \question[10] \textbf{Power of induction.}

  Consider the following algorithm to compute $a^b$, where $a$ and $b$ are some integers. 

\begin{minipage}{\linewidth}
\begin{algorithm}[H]
      \begin{algorithmic}[1]
        \Function{\text{Pow}}{$a, b$}
        \If{$b=0$}
            \State \Return 1
        \EndIf
        \If{$b$ is even}
            \State \Return $(\Call{Pow}{a, b/2})^2$
        \Else{
            \State \Return $a\cdot (\Call{Pow}{a, (b-1)/2})^2$
        }
        \EndIf
        \EndFunction
      \end{algorithmic}
\end{algorithm}
\end{minipage}

Prove that the algorithm is correct by induction. 
\begin{solution}
Proof:
\\Let $a,b$ be an arbitrary integer($b$ is positive). \\
We prove the correctness of the algorithm by induction on $b$.\\
\textbf{Base case}: $b=0$. The algorithm returns 1 $= a^0$, which is correct.\\\\
\textbf{Inductive step}: Assume that the algorithm is correct for all $b \leq k$, where $k \geq 0$. \\Now we show that the algorithm is correct for $b = k+1$.\\
There are only two cases to consider:
\\Case 1: $k+1$ is even, i.e. $k+1 = 2m$ for some integer $m$.\\
Then 
the algorithm returns $(\Call{Pow}{a, (k+1)/2})^2 = (\Call{Pow}{a, m})^2 = a^m \cdot a^m = a^{2m} = a^{k+1}$, which is correct.\\
\\Case 2: $k+1$ is odd, i.e. $k+1 = 2m+1$ for some integer $m$. So $k = 2m$.\\
Then
the algorithm returns $a\cdot (\Call{Pow}{a, (k+1-1)/2})^2 = a\cdot (\Call{Pow}{a, 2m/2})^2 = a\cdot (\Call{Pow}{a, m})^2 = a\cdot a^m \cdot a^m = a^{2m+1} = a^{k+1}$, which is correct.\\
Since Both cases are correct, we have proved the induction step.\\

Therefore, by induction, the algorithm is correct for all $b \geq 0$.
\end{solution}


  \question \textbf{Pigeons and the Contra-.}
  
  Recall the Pigeonhole Principle, which states ``If $n$ pigeons are placed into $m$ pigeonholes, where $n>m$, then at least one pigeonhole contains more than one pigeon." 
  
  Prove the Pigeonhole Principle using

    \begin{parts}
  \part[5] proof by contradiction, and

  \part[5] proof by contrapositive.

  \end{parts}

  \begin{solution}
  \begin{parts}
  \part Proof by contradiction:\\
  Suppose for sake of contradiction that $n$ pigeons are placed into $m$ pigeonholes, 
  (where $n>m$) with no pigeonhole contains more than one pigeon.\\
  Then each pigeonhole contains at most one pigeon.\\
  Since there are $m$ pigeonholes, there can at most be $m$ pigeons, so the total number of pigeons $n \leq m$, which contradicts the assumption that $n>m$. 
  \\Therefore the Pigeonhole Principle is true.\\
  \part Proof by contrapositive:
  \\The contrapositive of the statement is: For the $n$ pigeons placed into $m$ pigeonholes, if no pigeonhole contains more than one pigeon, then $n \leq m$.\\
  Suppose that $n>m$. Then there are more pigeons than pigeonholes.\\
  Since no pigeonhole contains more than one pigeon, the number of pigeons in every hole is less or equal to 1.\\
  Since there are $m$ holes, the total number of pigeons $n \leq m$.\\
  Then we have proved the contrapositive of the statement. So the Pigeonhole Principle is true.\\

  \end{parts}
  \end{solution}

\question[15] \textbf{Potential method.}

Alice is playing a \texttt{FactorFinding} game with herself. The integer factorization is not exciting enough so she decides to consider another number system.

    Alice thinks about the ``complex integers'', i.e.~$a+bi$ where $a,b$ are integers and $i^2=-1$. For example, 
    \begin{itemize}
        \item addition: $(1+2i)+(3+4i)=4+6i$;
        \item multiplication: $(1+2i)(3+4i)=3-8+(6+4)i=-5+10i$
    \end{itemize}
    She then plays the game as follows.
     Alice starts with an arbitrary ``complex integer'' $a_1+b_1 i$ and $k=1$. Alice tries to find  $a_{k+1}+b_{k+1} i$ that divides $a_k+b_k i$ which means there exist two integers $c$ and $d$ such that $(a_{k+1}+b_{k+1}i)(c+di)=a_k+b_ki$.  If $|c|+|d|\leq1$, then the game terminates. Otherwise, she increments $k$ and repeats this step. 

     Note: The initial complex number $a_1 + b_1i \neq 0$; that is, at least one of $a_i$ and $b_i$ must be non-zero.

     
    Question: Can Alice continue the game forever? If yes, give an example infinite run. If not, prove it by the potential-function method.

    \textbf{Hint:} Try the potential function $\Phi(a+bi)=a^2+b^2$.
    
    \begin{solution}
    Alice cannot continue the game forever.\\
    Using the potential function $\Phi(a+bi)=a^2+b^2$, we can prove that the game will terminate.\\
    \\First we claim that: for all complex number $x = a_1+b_1i, y= a_2+b_2i, z=a_3+b_3i$, if $xy = z$, then $|x| \cdot |y| = |z|$.\\
    Proof: $|x| \cdot |y| = \sqrt{a_1^2+b_1^2} \cdot \sqrt{a_2^2+b_2^2} = \sqrt{(a_1^2+b_1^2)(a_2^2+b_2^2)} = \sqrt{a_1^2a_2^2 + b_1^2b_2^2 + a_1^2b_2^2 + a_2^2b_1^2} \\= \sqrt{(a_1a_2-b_1b_2)^2+(a_1b_2+a_2b_1)^2} = \sqrt{a_3^2 + b_3^2} = |z|$\\

    Then let $a_k+b_ki$ be the complex number at the $k$-th step, and let $c+di$ be the next possible factor.\\
    Case 1: $|c|+|d|\leq1$. Then the game terminates at $k+1$-th step.\\
    Case 2: $|c|+|d| > 1$. Since $c, d$ are integers, $|c|+|d|\geq2$, so $|c| \geq 1$ and $|d| \geq 1$.\\
    Then the norm of $c+di$ = $\sqrt{c^2+d^2} = \sqrt{|c|^2+|d|^2} \geq \sqrt{2}$, therefore $c^2 + d^2 \geq 2$.\\
    So by our first claim, $\Phi(a_{k}+b_{k}i) = (c^2 +d^2) \cdot \Phi(a_{k+1} + b_{k+1}i) \geq 2 \Phi(a_{k+1} + b_{k+1}i)$.\\
    \\
    Then $\Phi(a_{k+1}+b_{k+1}i) \leq {1\over2} \Phi(a_k+b_ki)$ for any step $k$.\\
    Suppose $\Phi(a_{1}+b_{1}i) = a$ for some arbitrary positive integer $a$. Then $\Phi(a_{k}+b_{k}i) \leq {1\over2^k} a$ for any step $k$. by the property of integer(real numbers), there must exist some $k_0\in \mathbb{Z}$ such that $2^{k_0} \leq a < 2^{k_0 + 1}$. So consider $k = k_0 + 1$, then we have $\Phi(a_{k}+b_{k}i) \leq 1$.\\
    Therefore the game will terminate.

    \end{solution}   

  \question \textbf{Master theorem.}
  
  Consider the recurrence \[T(n) = \sqrt{n} \cdot T(\sqrt{n}) + O(n).\]
    \begin{parts}
    \part[5] Explain why the master theorem cannot be applied \emph{directly} to give a closed form for $T(n)$.

    \begin{solution}
    Because the master theorem requires $T(n) = aT({n\over b}) + O(n^d)$ where $a$ is a constant and the size of subproblems is divided linearly. But here the size of subproblems is divided by $\sqrt{n}$ and the coefficient of $T(\sqrt{n})$ is not a constant.
    \end{solution}
    
    \part[5] Define $S(n) = T(n)/n$. Using substitution, write a recurrence for $S(n)$.
    \begin{solution}
    Since $T(n) = \sqrt{n} \cdot T(\sqrt{n}) + O(n)$ and $S(n) = {T(n)\over n} = {\sqrt{n} \cdot T(\sqrt{n}) + O(n)\over n} = {T(\sqrt{n})\over \sqrt{n}} + {O(n)\over n} = S(\sqrt{n}) + O(1)$.\\
    So $S(n) = S(\sqrt{n}) + O(1)$.
    \end{solution}
    
    \part[5] Let $n=2^m$ and define \(R(m) = S(2^m) = S(n)\). Using this substitution, write a recurrence for \(R(m)\).  
    \textbf{Hint: } You may need to use that $\sqrt{n} = \sqrt{2^m} = 2^{m/2}$ 
    
    \begin{solution}
     \(R(m) = S(2^m) = S(n)\) = \(S(\sqrt{n}) + O(1)\) = \(S(2^{m/2}) + O(1)\) = \(R(m/2) + O(1)\).
     \\Therefore, \(R(m) = R(m/2) + O(1)\).
    \end{solution}
    
    \part[5] Use the Master Theorem and the above recurrence to get an asymptotic expression for \(R(m)\), then use it to get asymptotic expressions for \(S(n)\) and finally \(T(n)\).
    
    \begin{solution}
    For \(R(n) = R(n/2) + O(1)\), we have \(k = 1, b = 2, d = 1\). Since \(b^d = 2^1 = 2 > 1\), we have \(R(n) = O(n)\).
    \\And then, since \(R(m) = S(2^n)\), we have \(S(n) = R(\log_2 n) = O(\log n)\) and \(T(n) = nS(n) = O(n\log n)\).
    \end{solution}
    \end{parts}

\question[15] \textbf{Divide and conquer algorithm.}

The matrix $H_t$ is a $n\times n$ matrix where $n=2^t$.
It is defined recursively, where $H_0 = (1)$, and in general,
\[
H_{t+1} = \left(
\begin{array}{rr}
H_t  & H_t\\
H_t  & - H_t
\end{array}
\right)
\]
For example,
\[
H_2 = \left(
\begin{array}{rrrr}
1 & 1 & 1 & 1 \\
1 & -1& 1 & -1\\
1 & 1 & -1 & -1\\
1 & -1 & -1 & 1
\end{array}
\right)
\]
Given a vector $x \in \mathbb{Z}^{n}$, there is a trivial algorithm that uses $O(n^2)$ operations to compute the matrix-vector product $H_t x$, by first computing the matrix $H_t$ and then computing its product with $x$.
Describe an algorithm that computes $H_t x$ in $O(n \cdot t)=O(n\log n)$ operations.

\textbf{Note: } Recall how matrix vector multiplication works. If we have a $n\times n$ matrix $A$ and a vector $x\in \mathbb{Z}^n$, their product is calculated by taking the dot product of $x$ with each row in $A$. (That is, if the row is $[a_1, a_2, ..., a_n]$ and $x = \begin{bmatrix}
       x_1 \\
       x_2 \\
       \vdots \\
       x_n
       \end{bmatrix}$, then the dot product is $a_1x_1 + a_2x_2 + ... + a_nx_n$). For example, here we have a $2\times 2$ matrix $A$ and a vector $x\in \mathbb{Z}^2$.
\begin{equation*}
       A x = 
       \begin{bmatrix}
       a & b \\
       c & d
       \end{bmatrix}
       \begin{bmatrix}
       x_1 \\
       x_2
       \end{bmatrix}
       = 
       \begin{bmatrix}
       ax_1 + bx_2 \\
       cx_1 + dx_2
       \end{bmatrix}
   \end{equation*}

   Moreover, instead of multiplying each individual row and column, we can also multiply matrices in blocks. For example, if we have a $4\times 4$ matrix $A$ and a vector $x\in \mathbb{Z}^4$, we can break $A$ into 4 smaller matrices, say $2\times2$ matrices $H, I, J, K$. We can also break $x$ into 2 vectors $\vec{x}_1$ and $\vec{x}_2$ each of size 2. This would give the following formulation of the product $Ax$.
    \begin{equation*}
       A x = 
       \begin{bmatrix}
       H & I \\
       J & K
       \end{bmatrix}
       \begin{bmatrix}
       \vec{x}_1 \\
       \vec{x}_2
       \end{bmatrix}
       = 
       \begin{bmatrix}
       H\vec{x}_1 + I\vec{x}_2 \\
       J\vec{x}_1 + K\vec{x}_2
       \end{bmatrix}.
   \end{equation*}

\begin{solution}
\begin{algorithm}[H]
Input: int $n$ and vector $\vec{x} \in \mathbb{Z}^{n}$, where $n=2^t$ for some $t\in \mathbb{Z}_{\geq 0}$; \\
Output: vector $\vec{y}\in \mathbb{Z}^{n}$
      \begin{algorithmic}[1]
        \Function{\text{lineartrans}}{$n, \vec{x}$}
        \If{$n=1$}
            \State \Return x
        \Else{
            \State $t = \log_2(n)$
            \State $\vec{x}_1$ = $\vec{x}[0:{n\over 2}-1]$
            \State $\vec{x}_2$ = $\vec{x}[{n\over 2}:n-1]$
            \State $\vec{y}_1$ = \Call{lineartrans}{${n\over 2}, \vec{x}_1$}
            \State $\vec{y}_2$ = \Call{lineartrans}{${n\over 2}, \vec{x}_2$}
            \State $y[0:{n\over 2}-1] = \vec{y}_1+\vec{y}_2$
            \State $y[{n\over 2}:n-1] = \vec{y}_1-\vec{y}_2$
            \State \Return y
        }
        \EndIf
        \EndFunction
      \end{algorithmic}
\end{algorithm}
This algorithm computes $H_t x$ in $O(n\log n)$ operations. Slicing $\vec x$ into $\vec{x_1}, \vec{x_2}$ and concatenating $vec{y}$ by $vec{y_1}$ and $\vec{y_2}$ takes linear time (within $2n$), so the recurrence relation for the algorithm is $T(n) = 2T(n/2) + O(n)$, which is $O(n\log n)$ by the Master Theorem.
\end{solution}

   

  \bonusquestion[10] \textbf{Extra credit:} \emph{You are not required to do this question to receive full credit on this assignment.}
  To receive the bonus points, you must typeset this \textbf{entire} assignment in \LaTeX, and draw a table with two columns that includes the \emph{name} (e.g., ``fraction'') and an \emph{example} of each of the following:
  \begin{itemize}
  \item fraction (using \texttt{\textbackslash frac}),
  \item less than or equal to,
  \item union of two sets,
  \item summation using Sigma ($\sum$) notation,
  \item the set of real numbers ($\R$); write a mathematically correct statement that applies to \emph{all} real numbers $x\in \R$.
  \end{itemize}
  
  \begin{solution}
    \begin{tabular}{|c|c|}
    \hline
    \textbf{name} & \textbf{example} \\
    \hline
    fraction & $\frac{1}{3} = \frac{2}{6}$ \\
    \hline
    less than or equal to & $2ab \leq a^2 + b^2$\\
    \hline
    union of two sets & $\{1,2\} \bigcup \{3,4\} = \{1,2,3,4\}$ \\
    \hline
    summation & $\sum_{i=1}^{10} i= 55 $\\
    \hline
    the set of real numbers & $\mathbb{R}.$ $\forall a\in \mathbb{R}$, $a \in \mathbb{Q}$ or $a \in \mathbb{R} \backslash \mathbb{Q}$.\\
    \hline
  \end{tabular}
  \end{solution} 
\end{questions}

\end{document}
