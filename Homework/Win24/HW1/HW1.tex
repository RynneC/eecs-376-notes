\documentclass[11pt,addpoints]{exam}
\usepackage{fullpage}
\usepackage{amsfonts, amsmath, amssymb, amsthm}
\usepackage{mathtools}
\usepackage{enumerate}
\usepackage{hyperref}

% in order to compile this file you need to get 'header.tex' from
% Canvas and change the line below to the appropriate file path
%%% theorems

\theoremstyle{plain}            % following are "theorem" style
\newtheorem{theorem}{Theorem}[section]
\newtheorem{lemma}[theorem]{Lemma}
\newtheorem{corollary}[theorem]{Corollary}
\newtheorem{proposition}[theorem]{Proposition}
\newtheorem{claim}[theorem]{Claim}
\newtheorem{fact}[theorem]{Fact}
\newtheorem{openproblem}[theorem]{Open Problem}

\theoremstyle{definition}       % following are def style
\newtheorem{definition}[theorem]{Definition}
\newtheorem{conjecture}[theorem]{Conjecture}
\newtheorem{example}[theorem]{Example}
\newtheorem{protocol}[theorem]{Protocol}
\newtheorem{exercise}[theorem]{Exercise}

\theoremstyle{remark}           % following are remark style
\newtheorem{remark}[theorem]{Remark}
\newtheorem{note}[theorem]{Note}
\newtheorem*{aside}{Aside}
\newtheorem*{gn}{Grading Note} % grading note

%%% special sets
\newcommand{\bit}{\ensuremath{\{0,1\}}}
\newcommand{\bitt}{\ensuremath{\{-1,1\}}}

% font for general-purpose algorithms
\newcommand{\algo}[1]{\ensuremath{\mathsf{#1}}}
% font for general-purpose computational problems
\newcommand{\problem}[1]{\ensuremath{\mathsf{#1}}}
% font for complexity classes
%\newcommand{\class}[1]{\ensuremath{\mathsf{#1}}}

\global\long\def\path{\mathrm{path}}
% asymptotics
%\DeclareMathOperator{\poly}{poly}
%\DeclareMathOperator{\polylog}{polylog}
\DeclareMathOperator{\negl}{negl}
\DeclareMathOperator{\Otil}{\tilde{O}}
\DeclareMathOperator{\Ostar}{O^*}

%%% "LEFT-RIGHT" PAIRS OF SYMBOLS

% inner product
\DeclarePairedDelimiter\inner{\langle}{\rangle}
% absolute value
\DeclarePairedDelimiter\abs{\lvert}{\rvert}
% a set
\DeclarePairedDelimiter\set{\{}{\}}
% parens
\DeclarePairedDelimiter\parens{(}{)}
% tuple, alias for parens
\DeclarePairedDelimiter\tuple{(}{)}
% square brackets
\DeclarePairedDelimiter\bracks{[}{]}
% rounding off
\DeclarePairedDelimiter\round{\lfloor}{\rceil}
% floor function
\DeclarePairedDelimiter\floor{\lfloor}{\rfloor}
% ceiling function
\DeclarePairedDelimiter\ceil{\lceil}{\rceil}
% length of some vector, element
\DeclarePairedDelimiter\length{\lVert}{\rVert}
% norm
\DeclarePairedDelimiter\norm{\lVert}{\rVert}
% length (as a string)
\DeclarePairedDelimiter\len{\lvert}{\rvert}
% bra-kets
\DeclarePairedDelimiter\bra{\langle}{\rvert}
\DeclarePairedDelimiter\ket{\lvert}{\rangle}
\newcommand{\braket}[2]{\ensuremath{\langle #1 \vert #2 \rangle}}
\newcommand{\ketbra}[2]{\ensuremath{\lvert #1 \rangle \langle #2 \rvert}}

%%% MATH OPERATORS
\DeclareMathOperator{\Var}{Var}
\DeclareMathOperator{\tr}{Tr}
\DeclareMathOperator{\supp}{supp}
\DeclareMathOperator{\var}{var}
\DeclareMathOperator{\vol}{vol}
\DeclareMathOperator{\sign}{sign}

%%% BLACKBOARD SYMBOLS

\newcommand{\C}{\ensuremath{\mathbb{C}}}
\newcommand{\D}{\ensuremath{\mathbb{D}}}
\newcommand{\F}{\ensuremath{\mathbb{F}}}
\newcommand{\G}{\ensuremath{\mathbb{G}}}
\newcommand{\J}{\ensuremath{\mathbb{J}}}
\newcommand{\N}{\ensuremath{\mathbb{N}}}
\newcommand{\Q}{\ensuremath{\mathbb{Q}}}
\newcommand{\R}{\ensuremath{\mathbb{R}}}
\newcommand{\T}{\ensuremath{\mathbb{T}}}
\newcommand{\Z}{\ensuremath{\mathbb{Z}}}
\newcommand{\QR}{\ensuremath{\mathbb{QR}}}

% sets in calligraphic type

\newcommand{\calD}{\ensuremath{\mathcal{D}}}
\newcommand{\calF}{\ensuremath{\mathcal{F}}}
\newcommand{\calG}{\ensuremath{\mathcal{G}}}
\newcommand{\calH}{\ensuremath{\mathcal{H}}}
\newcommand{\calI}{\ensuremath{\mathcal{I}}}
\newcommand{\calL}{\ensuremath{\mathcal{L}}}
\newcommand{\calN}{\ensuremath{\mathcal{N}}}
\newcommand{\calP}{\ensuremath{\mathcal{P}}}
\newcommand{\calS}{\ensuremath{\mathcal{S}}}
\newcommand{\calX}{\ensuremath{\mathcal{X}}}
\newcommand{\calY}{\ensuremath{\mathcal{Y}}}

% matrices and vectors

\newcommand{\matA}{\ensuremath{\mathbf{A}}}
\newcommand{\matB}{\ensuremath{\mathbf{B}}}
\newcommand{\matC}{\ensuremath{\mathbf{C}}}
\newcommand{\matD}{\ensuremath{\mathbf{D}}}
\newcommand{\matE}{\ensuremath{\mathbf{E}}}
\newcommand{\matF}{\ensuremath{\mathbf{F}}}
\newcommand{\matG}{\ensuremath{\mathbf{G}}}
\newcommand{\matH}{\ensuremath{\mathbf{H}}}
\newcommand{\matI}{\ensuremath{\mathbf{I}}}
\newcommand{\matJ}{\ensuremath{\mathbf{J}}}
\newcommand{\matK}{\ensuremath{\mathbf{K}}}
\newcommand{\matL}{\ensuremath{\mathbf{L}}}
\newcommand{\matM}{\ensuremath{\mathbf{M}}}
\newcommand{\matN}{\ensuremath{\mathbf{N}}}
\newcommand{\matO}{\ensuremath{\mathbf{O}}}
\newcommand{\matP}{\ensuremath{\mathbf{P}}}
\newcommand{\matQ}{\ensuremath{\mathbf{Q}}}
\newcommand{\matR}{\ensuremath{\mathbf{R}}}
\newcommand{\matS}{\ensuremath{\mathbf{S}}}
\newcommand{\matT}{\ensuremath{\mathbf{T}}}
\newcommand{\matU}{\ensuremath{\mathbf{U}}}
\newcommand{\matV}{\ensuremath{\mathbf{V}}}
\newcommand{\matW}{\ensuremath{\mathbf{W}}}
\newcommand{\matX}{\ensuremath{\mathbf{X}}}
\newcommand{\matY}{\ensuremath{\mathbf{Y}}}
\newcommand{\matZ}{\ensuremath{\mathbf{Z}}}
\newcommand{\matzero}{\ensuremath{\mathbf{0}}}

\newcommand{\veca}{\ensuremath{\mathbf{a}}}
\newcommand{\vecb}{\ensuremath{\mathbf{b}}}
\newcommand{\vecc}{\ensuremath{\mathbf{c}}}
\newcommand{\vecd}{\ensuremath{\mathbf{d}}}
\newcommand{\vece}{\ensuremath{\mathbf{e}}}
\newcommand{\vecf}{\ensuremath{\mathbf{f}}}
\newcommand{\vecg}{\ensuremath{\mathbf{g}}}
\newcommand{\vech}{\ensuremath{\mathbf{h}}}
\newcommand{\veck}{\ensuremath{\mathbf{k}}}
\newcommand{\vecm}{\ensuremath{\mathbf{m}}}
\newcommand{\vecp}{\ensuremath{\mathbf{p}}}
\newcommand{\vecq}{\ensuremath{\mathbf{q}}}
\newcommand{\vecr}{\ensuremath{\mathbf{r}}}
\newcommand{\vecs}{\ensuremath{\mathbf{s}}}
\newcommand{\vect}{\ensuremath{\mathbf{t}}}
\newcommand{\vecu}{\ensuremath{\mathbf{u}}}
\newcommand{\vecv}{\ensuremath{\mathbf{v}}}
\newcommand{\vecw}{\ensuremath{\mathbf{w}}}
\newcommand{\vecx}{\ensuremath{\mathbf{x}}}
\newcommand{\vecy}{\ensuremath{\mathbf{y}}}
\newcommand{\vecz}{\ensuremath{\mathbf{z}}}
\newcommand{\veczero}{\ensuremath{\mathbf{0}}}
\newcommand{\vecone}{\ensuremath{\mathbf{1}}}

%%% misc

\newcommand{\eps}{\varepsilon}

%%% for homework and section notes

\newcommand{\commonheader}[2]{
    \pagestyle{headandfoot}
    \setlength{\headheight}{26pt}
    \setlength{\headsep}{16pt}

    \header
        {\small{\textbf{EECS 376: Foundations of Computer Science}} \\ \footnotesize{\textbf{University of Michigan, Spring 2024}}}
        {#1}
        {#2}

    \firstpageheadrule
    \runningheadrule

    \footer
        {}
        {\thepage}
        {}
}

\newcommand{\hwheader}{
    \commonheader
        {\Large \textbf{Homework \hwnum}}
        {\small \textbf{Due 8:00pm, \duedate\\ {\tiny(accepted until 9:59 pm, no credit after)}}}
}

\newcommand{\hwslnheader}{
    \commonheader
    	{}
        {\Large \textbf{Solutions to Homework \hwnum}}
    \printanswers
}

\newcommand{\notesheader}{
    \commonheader
    	{}
        {\Large \textbf{Discussion Notes \sectionnum}}
}

\newcommand{\practiceheader}{
    \commonheader
    	{}
        {\Large \textbf{Discussion Worksheet \sectionnum}}
}

\newcommand{\practiceslnheader}{
    \commonheader
    	{}
        {\Large \textbf{Solutions to Discussion Worksheet \sectionnum}}
}

\newcommand{\reviewheader}{
    \commonheader 
    \smallskip
    	{}
        {\Large \textbf{Midterm Review Notes}}
}

\newcommand{\hwpreface}{

\noindent This homework has \numquestions\ questions, for a total of \numpoints\ points and \numbonuspoints\ extra-credit points.

\noindent Unless otherwise stated, each question requires \emph{clear}, \emph{logically correct}, and \emph{sufficient} justification to convince the reader.

\noindent For bonus/extra-credit questions, we will provide very limited guidance in office hours and on Piazza, and we do not guarantee anything about the difficulty of these questions.
 
\noindent We strongly encourage you to typeset your solutions in \LaTeX.

\noindent If you collaborated with someone, you must state their name(s). You must \emph{write your own solution} for all problems and \emph{may not use any other student’s write-up}.
}

\newcommand{\hint}[1]{
\emph{Hint}: #1
}
\newcommand{\bonus}{
\emph{Optional bonus}:
}
\newcommand{\extracredit}{
\emph{Extra credit}: 
}


\newcommand{\hwnum}{1}
\newcommand{\duedate}{January 17}
\usepackage{xcolor}

\hwheader   % header for homework

% Comment the following line in order to hide solutions.
% Uncomment the line to show solutions written inside of
% LaTeX solution environments like:
%   \begin{solution}
%     My solution.
%   \end{solution}.
\printanswers

\begin{document}

\hwpreface

\pointsinmargin
\pointpoints{pt}{pts}
\bonuspointpoints{EC pt}{EC pts}
\marginpointname{ \points}
\marginbonuspointname{ \bonuspoints}

\begin{questions}

  \addtocounter{question}{-1}
  \question[0] \textbf{Before you start; before you submit.}
  
  \begin{parts}
    \part Carefully read \href{https://drive.google.com/drive/u/1/folders/1fHV5RQgjvt0rWCTYv2EExI0M3cVDyi6G}{Handout 1} before starting this assignment, and apply it to the solutions you submit.
    
    \part If applicable, state the name(s) and uniqname(s) of your collaborator(s).

    \begin{solution}
    
      
    \end{solution}
  \end{parts}

  \question \textbf{Properties of logarithms.}
  
  The objective of this problem is to recall some important properties of logarithms that we'll use throughout the course.
  First recall the following properties of exponentiation for any real numbers $b, x, y$, assuming that $b \neq 0$:
  \begin{description}
  \item[P1:] $(b^x)^y = b^{xy}$
  \item[P2:] $b^x = b^y$ if and only if $x = y$, additionally assuming that $b \neq 1$.
  \end{description}

  Now, for positive real numbers $b \neq 1$ and~$x$, the definition of the base-$b$ logarithm is:
  \begin{description}
  \item[D1:] $y = \log_b x$ is the unique real number for which $b^y = x$.
    That is, it satisfies the identity $b^{\log_{b} x} = x$.
    (Uniqueness of this value follows directly from property P2.)
  \end{description}
  
  Combining D1 with P2, we see that $x=y$ if and only if $\log_{b} x = \log_{b} y$ (assuming that $b,x,y$ are positive and $b \neq 1$).
  So, equality is preserved by taking logarithms or exponentials (with the same base) on both sides of an equation, assuming the hypotheses are met.

  Prove each of the following properties of logarithms, for any positive real numbers (\emph{not necessarily integers}) $a, b, c, n$ where $b, c \neq 1$.
  Cite definition D1 and properties P1 and P2 where you use them; you may also use (with citation) the results of previous parts of the problem. 
  Do not use any other assumptions about, or properties of, logarithms and exponentials.

  For reference, we have provided a solution to the first part, which you can use as a model for later solutions.
  
  \begin{parts}
    \part[0] $\log_{b} (a^{n}) = n \log_{b} a$.
    (Be careful: $n$ is not necessarily an integer!)

    \begin{solution}
      Notice that~$n$ is not necessarily an integer; the problem stipulates only that it is a positive real number.

      By P2, the statement we want to prove is equivalent to $b^{\log_{b} (a^{n})} = b^{n \log_{b} a}$ (since $b \neq 1$).
      By D1, the left-hand side is equal to $a^{n}$, so it suffices to prove that $a^{n} = b^{n \log_{b} a}$.
      Indeed,
      \[ b^{n \log_{b} a} = (b^{\log_{b} a})^{n} = a^{n} , \]
      where the first equality is by P1, and the second equality is by D1.

      (As a remark, this proof does not rely on the assumption that~$n$ is positive, so the statement holds even more generally for any real number~$n$.
      However, later parts require~$n$ to be positive.)
    \end{solution}
                
    \part[5] $\log_{b} a = \frac{\log_{c} a}{\log_{c} b}$.
    (This is sometimes called the ``change-of-base theorem,'' because it lets us change the base of the logarithm by dividing by an appropriate factor.
    Note that $\log_{c} b \neq 0$ because $b \neq 1$, so the division is well defined.)

    \begin{solution}
      
    \end{solution}
    
    \part[5] The ratio $\frac{\log_{b} n}{\log_{c} n}$ (for $n \neq 1$) depends only on~$b$ and~$c$, not~$n$.

    This implies that we can often omit the (constant) base when writing logarithms in asymptotic notation, e.g., we can write $O(\log n)$ instead of $O(\log_2 n)$, because $O(\cdot)$ hides constant factors that do not depend on~$n$.
    However, we cannot omit the base when the logarithm appears in the exponent, as in $O(3^{\log_2 n})$.

    \begin{solution}
      
    \end{solution}
  \end{parts}
  
  \question \textbf{Practice with asymptotics (``big-Oh, big-Omega, big-Theta'').}

  Let $f(n)$ and $g(n)$ be positive functions.
  Recall that $f(n) = O(g(n))$ if there exist constants $c, n_0 > 0$ such that $f(n) \leq c \cdot g(n)$ for all $n \geq n_0$.

  In words, we often phrase this as: ``$f(n)$ is \emph{at most} (or: is \emph{upper-bounded by}) a \emph{constant multiple} of $g(n)$, for \emph{all large enough}~$n$.''
  (``All large enough~$n$'' refers to the clause ``all $n \geq n_{0}$'', i.e., all~$n$ beyond some fixed threshold.)

  A \emph{sufficient} condition for $f(n) = O(g(n))$ is that $\lim_{n \to \infty} \frac{f(n)}{g(n)}$ exists and is finite.
  However, this condition is not \emph{necessary}; it may be that $f(n)=O(g(n))$ even if the limit does not exist.

  Similarly, $f(n) = \Omega(g(n))$ if there exist constants $c, n_{0} > 0$ such that $f(n) \geq c \cdot g(n)$ for all $n \geq n_{0}$.
  \begin{itemize}
  \item In words: ``$f(n)$ is \emph{at least} (or: is \emph{lower-bounded by}) a (positive) constant multiple of $g(n)$, for all large enough~$n$.''
  \item A sufficient condition is that $\lim_{n \to \infty} \frac{f(n)}{g(n)}$ exists and is nonzero (this includes the case where the limit is infinite).
  \item It can be seen that $f(n) = \Omega(g(n))$ is equivalent to $g(n) = O(f(n))$, since $f(n) \geq c \cdot g(n)$ if and only if $g(n) \leq (1/c) \cdot f(n)$, and~$c$ is a positive constant if and only if $1/c$ is one.
  \end{itemize}
  
  Finally, $f(n) = \Theta(g(n))$ if both $f(n) = O(g(n))$ and $f(n) = \Omega(g(n))$.

  For the following pairs of functions, state with justification whether or not each of the following hold: $f(n) = O(g(n))$, $f(n) = \Omega(g(n))$, $f(n) = \Theta(g(n))$.

  \hint{You might find L'H\^{o}pital's Rule useful: it says that $\lim_{n \to \infty} \frac{f(n)}{g(n)} = \lim_{n \to \infty} \frac{f'(n)}{g'(n)}$ when the latter limit exists, where~$f'$ and~$g'$ are respectively the derivatives of~$f$ and~$g$.}

  \begin{parts}
    \part[5] $f(n) = 3n^2 + 10n + 6$, $g(n) = n^2$.

    (Optional challenge: generalize to similar-looking $f$ and $g$, beyond specific constant coefficients or exponents.)
    
    \begin{solution}
      
    \end{solution}

    \part[5] $f(n) = \log_{2}(n^2)$, $g(n) = \log_{10}(n^{10})$.

    (Optional challenge: generalize to similar-looking $f$ and $g$, beyond specific constant bases and exponents.)
    
    \begin{solution}
 
    \end{solution}

    \part[5] $f(n) = n^{5} \cdot 2^{3n}$, $g(n) = 3^{2n}$.
    
    (Optional challenge: generalize to similar-looking $f$ and $g$, beyond specific constants in the bases and exponents.)

    \begin{solution}
      
    \end{solution}

  \end{parts}

  \question \textbf{Comparing asymptotic running times.}

  Suppose that Algorithm X has running time $T_{X}(n) = O(n^{2})$, Algorithm Y has running time $T_{Y}(n) = \Theta(n \log n)$, and Algorithm Z has running time $T_{Z}(n) = \Theta(n)$.

  As usual, all running times are stated in terms of the \emph{worst case} for inputs of size~$n$.
  That is, $T_{X}(n)$ is the \emph{maximum} number of steps for which X runs, taken over all inputs of size~$n$ (and similarly for $T_{Y}, T_{Z}$).

  \begin{parts}
    \part[10] Consider the claim: ``on every input, Y runs faster than~X.''
    Clearly explain why this claim is (i)~not necessarily true, and (ii)~not necessarily false.
    To do this, provide concrete running times that are consistent with the stated asymptotic bounds, and which make the claim true or false, as appropriate.

    \begin{solution}
     
    \end{solution}

    \part[10] Consider the claim: ``for all large enough~$n$, there is an  input of size~$n$ for which~Y runs faster than~X.''
    Clearly explain why this claim is (i)~not necessarily true, and (ii)~not necessarily false.

    \begin{solution}
 
    \end{solution}

    \part[10] Consider the claim: ``for all large enough~$n$, there is an input of size~$n$ for which~Z runs faster than~Y.''
    Is this claim necessarily true, necessarily false, or not necessarily either?
    Clearly justify your answer.

    \begin{solution}
      
    \end{solution}
  \end{parts}

  \question[10] \textbf{Homework solver!}

Consider the following homework solver:
   
\begin{verbatim}
   function SOLVE-HW(X):            // X is a non-negative integer
      for i = 1 to floor(X^(1/3)):  // floor of cube root of X
         print(`hey grader, give me ' + X + ` points')
      return X
\end{verbatim} 
   
  Give an asymptotic (big-O) bound on the worst-case number of characters printed by the program as a function of the \emph{size} of the (non-negative integer) input X, written in binary.
  Make this bound as tight as possible.
  Is the program ``efficient,'' as defined by this course?
   
  (Note: our providing this program does not constitute a guarantee of any particular lower bound on the number of points you should expect to receive if you choose to run it for your own homeworks.)
   
  \begin{solution}
   
  \end{solution}

  \question[10] \textbf{Practice with induction.}
  
  Use induction to prove that for any integer $n \geq 0$, the number of binary strings having length \emph{at most}~$n$ (including length zero) is exactly $2^{n+1}-1$.

  (You may use the fact that there are exactly $2^{k}$ binary strings of length exactly~$k$, which was proven in discussion.)
  
  \begin{solution}
   
  \end{solution}
    
  \question[10] \textbf{Representing data.}

  Suppose you want to unambiguously represent the elements of $\set{0, 1, \ldots, n-1}$ (for some positive integer $n$) using ternary strings of a certain length~$k$, i.e., strings with exactly $k$ characters from the set $\set{a,b,c}$.
  What is the \emph{smallest} possible choice of $k$ for which this is possible?
  
  For example, to represent the elements of $\{0,1,2,3,4,5\}$, we can use the ternary strings $aa$, $ab$, $ac$, $ba$, $bb$, and $bc$ of length $k=2$.
  This is the smallest length possible to uniquely represent these elements, because there are only $3$ ternary strings of length $k=1$.
  
  \begin{solution}
   
  \end{solution}
  
  \question \textbf{Proofs: direct, and by contrapositive.}

  Recall that a real number is \emph{rational} if it can be written as $p/q$ for some integers $p,q$, where $q \neq 0$.
  
  \begin{parts}
    \part[5] Given a direct proof that if $r \neq 0$ is rational, then $1/r$ is rational.
    What is the contrapositive of this statement?
    (There is more than one valid way to write it.)

    \begin{solution}
      
    \end{solution}
    
    \part[5] Give a proof by contrapositive (\emph{not} by contradiction!) that if $r \neq 0$ is rational and~$x$ is irrational, then $x \cdot r$ is irrational.
    First state the contrapositive (there is more than one valid way to write it), then prove that directly.
    Be sure to explicitly say where you use the hypothesis that $r \neq 0$ (because the statement is false without it!).

    \begin{solution}
      
    \end{solution}
    
    \part[5] Prove that any nonzero rational number may be written as the product of two irrational numbers.
    You may use the fact that $\sqrt{2}$ is irrational without proof.
    Again, explicitly say where you use the hypothesis that the number in question is nonzero (because the statement it false without it!).

    \begin{solution}
      
    \end{solution}
  \end{parts}

  \bonusquestion[10] \textbf{Extra credit:} \emph{You are not required to do this question to receive full credit on this assignment.}
  To receive the bonus points, you must typeset this \textbf{entire} assignment in \LaTeX, and draw a table with two columns that includes the \emph{name} (e.g., ``fraction'') and an \emph{example} of each of the following:
  \begin{itemize}
  \item fraction (using \texttt{\textbackslash frac}),
  \item less than or equal to,
  \item union of two sets,
  \item summation using Sigma ($\sum$) notation,
  \item the set of real numbers ($\R$); write a mathematically correct statement that applies to \emph{all} real numbers $x\in \R$.
  \end{itemize}
  
  \begin{solution}
    
    \end{solution}

\end{questions}

\end{document}
